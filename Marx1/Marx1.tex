\documentclass{ctexart}

\usepackage[top=2cm, bottom=2cm, left=2.5cm, right=2.5cm]{geometry}
\usepackage{graphicx}
\graphicspath{{D:/code/LATEX/Marx}}
\usepackage{amsmath,amssymb,amsfonts}
\usepackage{array}
\usepackage{booktabs}
\usepackage{float}
\usepackage{pifont}
\usepackage{enumitem}
\vfuzz=100pt  % 垂直方向容忍度
\hfuzz=100pt  % 水平方向容忍度  
\vbadness=10000
\hbadness=10000
\overfullrule=0pt  % 不标记过满行

\begin{document}

\section{导论}

1.什么是马克思主义

马克思主义是由马克思、恩格斯创立并为后继者所不断发展的科学理论体系,是关于自然、社会和人类思维发展一般规律的学说,是关于社会主义必然代替资本主义、最终实现共产主义的学说,是关于无产阶级解放、全人类解放和每个人自由而全面发展的学说,是无产阶级政党和社会主义国家的指导思想,是指引人民创造美好生活的行动指南。\\
组成部分:\begin{itemize}
    \item 马克思主义哲学
    \item 马克思主义政治经济学
    \item 科学社会主义
\end{itemize}
发展历程:\begin{itemize}
    \item 19世纪由马克思、恩格斯创立\[\begin{cases}
        \text{社会根源:19世纪资本主义矛盾激化(如贫富分化、经济危机)}\\
        \text{阶级基础:无产阶级斗争走向自觉(如三大工人运动)}\\
        \text{思想渊源:批判继承德国古典哲学、英国古典政治经济学、英法空想社会主义}\\
        \text{科学基础:19世纪自然科学三大发现(细胞学说、能量守恒定律、进化论)}
    \end{cases}\]
    \item 列宁主义:提出社会主义革命可能在一国或数国首先胜利
    \item 马克思主义中国化时代化\[\begin{cases}
        \text{毛泽东思想:开创农村包围城市道路,实现马克思主义中国化的第一次飞跃}\\
        \text{中国特色社会主义理论体系:包括邓小平理论、“三个代表”重要思想、科学发展观,实现新的飞跃}\\
        \text{习近平新时代中国特色社会主义思想:是当代中国马克思主义、21世纪马克思主义}
    \end{cases}\]
\end{itemize}

2.马克思主义的基本特征:科学性、人民性、实践性、发展性;\\
马克思主义的当代价值:是观察世界变化的认识工具,是指引当代中国发展的行动指南,是引领人类社会进步的科学真理

\section{世界的物质性及发展规律}

1.哲学的基本问题及派别:\begin{itemize}
    \item 世界的本原的第一性问题:唯物主义(物质第一性)和唯心主义(意识第一性)
    \item 思维和存在的同一性问题:可知论(思维能认识存在)和不可知论(思维不一定能认识存在)
    \item 是否承认矛盾是事物发展的动力:辩证法(联系、发展、全面)和形而上学(孤立、静止、片面)
\end{itemize}
马克思主义物质观:物质是客观实在,不依赖于意识并能被意识反映(可知论+唯物论+辩证法)

2.物质

物质是运动的,运动是绝对的(静止是相对的);物质世界是运动变化的,世界一方面分为自然界与人类社会,另一方面分为客观世界与主观世界;人类实践促使世界分化为自然界与人类社会、客观世界与主观世界,实践也是二者统一的基础。

3.物质与意识的辩证关系\begin{itemize}
    \item 意识的起源:从自然界演化到社会劳动的产物
    \item 意识的本质:客观世界的主观映象
    \item 物质与意识的辩证关系:物质决定意识,意识依赖于物质并反作用于物质
\end{itemize}

4.主观能动性与客观规律性的统一\begin{itemize}
    \item 尊重客观规律是发挥主观能动性的前提
    \item 发挥主观能动性是认识和利用规律的必要条件
    \item 实践是二者统一的桥梁
\end{itemize}

5.联系:指事物内部各要素之间和事物之间相互影响、相互制约和相互作用的关系。\\
特征:\begin{itemize}
    \item 客观性: 联系是事物本身固有的,不以人的意志为转移
    \item 普遍性:任何事物内部、事物之间、整个世界都处于普遍联系之中
    \item 多样性:内部/外部、直接/间接、必然/偶然、本质/非本质等
    \item 条件性:联系依赖于具体条件,条件可以改变,但必须尊重客观规律
\end{itemize}
联系构成运动、变化和发展;运动是一切形式的变化,变化的趋势是发展,发展的实质是新事物的产生和旧事物的灭亡;新事物是指合乎历史前进方向、具有远大前途的东西。

6.\textbf{对立统一规律}

矛盾是反应事物内部或事物之间对立统一关系的哲学范畴。

矛盾的基本属性:\begin{itemize}
    \item 同一性:矛盾双方相互依存、相互贯通(有条件的、相对的)
    \item 斗争性:矛盾双方相互排斥、相互分离(无条件的、绝对的)
    \item 普遍性:矛盾存在于一切事物中,存在于一切事物发展过程的始终
    \item 特殊性:各个具体事物的矛盾、每一个矛盾各方面在不同阶段上各有特点
\end{itemize}

矛盾的普遍性、特殊性对立统一;矛盾分主要矛盾(支配地位、决定作用)和次要矛盾(从属地位、次要作用)

矛盾的解决:\begin{itemize}
    \item 一方克服另一方
    \item 融为一体(发展为新的矛盾双方)
    \item 同归于尽
\end{itemize}

7.量变质变规律

事物包括质(内在规定性)、量(数量规定性)、度(保持质的数量界限)三方面的规定性;量变是保持质的不显著变化,质变是一种质向另一种质的飞跃,量变引起质变,界限是是否超过度。

量变与质变的辩证关系:\begin{itemize}
    \item 量变是质变的必要准备
    \item 质变是量变的必然结果
    \item 量变质变相互渗透
\end{itemize}

8.否定之否定规律

事物内部都存在着肯定因素(维持现存事物存在)和否定因素(促使现存事物灭亡)。事物的辩证否定经历“肯定——否定——否定之否定”三个阶段,呈现周期性;每一次否定都是质变。

唯物辩证法的否定观科学揭示了否定的深刻内涵:\begin{itemize}
    \item 否定是事物的自我否定、自我发展,是事物内部矛盾运动的结果
    \item 否定是事物发展的环节
    \item 否定是新旧事物联系的环节
    \item 辩证否定的实质是“扬弃”,即新事物对旧事物既批判又继承,既克服其消极因素又保留其积极因素
\end{itemize}

9.联系与发展的五对范畴
\[\begin{cases}
    \text{内容与形式:从构成要素和表现方式上反映事物}\\
    \text{本质与现象:揭示事物内在联系和外在表现}\\
    \text{原因与结果:揭示事物引起和被引起关系}\\
    \text{必然与偶然:揭示事物产生、发展和衰亡过程中的不同趋势}\\
    \text{现实与可能:反映事物的过去、现在和将来关系}
\end{cases}\]

10.唯物辩证法是认识世界和改造世界的根本方法\begin{itemize}
    \item 唯物辩证法本质上是批判的和革命的
    \item 唯物辩证法是客观辩证法与主观辩证法的统一\[\begin{cases}
        \text{客观辩证法是指客观事物或客观存在的辩证法}\\
        \text{主观辩证法是客观辩证法在人的思维中的反映}
    \end{cases}\]
    \item 唯物辩证法是科学的认识方法\[\begin{cases}
        \text{既要符合客观辩证法,又有其固有的辨证运动规律}\\
        \text{矛盾分析方法:分析特殊性}\\
        \text{坚持问题导向,问题是事物矛盾的表现形式}
    \end{cases}\]
    \item 学习唯物辩证法\[\begin{cases}
        \text{辩证思维能力}\begin{cases}
            \text{归纳与演绎}\\
            \text{分析与综合}\\
            \text{抽象与具体}\\
            \text{逻辑与历史}
        \end{cases}\\
        \text{历史思维能力}\\
        \text{系统思维能力}\\
        \text{战略思维能力}\\
        \text{底线思维能力}\\
        \text{创新思维能力}
        \end{cases}\]
\end{itemize}

\section{实践与认识及其发展规律}

1.\textbf{科学的实践观及其意义}\begin{itemize}
    \item 克服了旧唯物主义的缺陷
    \item 创立了能动的反映论,揭示了实践对认识的决定性作用
    \item 首次揭示社会生活的实践本质,为唯物史观奠定基础
    \item 提供了认识世界和改造世界的思想方法
\end{itemize}

2.实践\begin{itemize}
    \item 本质:人类能动地改造世界的社会性物质活动
    \item 基本特征\[\begin{cases}
            \text{客观实在性}\\
            \text{自觉能动性:实践是有目的、有意识的活动}\\
            \text{社会历史性:实践是在社会关系中进行的,实践受历史条件制约}
        \end{cases}\]
    \item 实践的基本结构:实践主体(人)、实践客体(对象)、实践中介(工具、方法)
    \item 基本形式\[\begin{cases}
            \text{物质生产实践}\\
            \text{社会政治实践}\\
            \text{科学文化实践}
        \end{cases}\]
    \item \textbf{对认识的决定作用}:实践是认识的基础\[\begin{cases}
            \text{实践是认识的来源}\\
            \text{实践是认识发展的动力}\\
            \text{实践是认识的目的}\\
            \text{实践是检验真理的唯一标准}
        \end{cases}\]
\end{itemize}

3.认识\begin{itemize}
    \item 认识的本质:实践基础上的主体对客体的能动反映\[\begin{cases}
            \text{唯物主义:反映论(物→感觉→思想)}\\
            \text{唯心主义:先验论(思想→感觉→物)}\\
            \text{旧唯物主义:以感性直观为基础}\\
            \text{辩证唯物主义:引入实践的观点和辩证法}
        \end{cases}\]
    \item 认识的发展过程\[\begin{cases}
            \text{第一次飞跃:从实践到认识}\begin{cases}
                \text{感性认识:低级阶段,形式为感觉、知觉、表象,具直接性、具体性}\\
                \text{理性认识:高级阶段,形式为概念、判断、推理,具间接性、抽象性}\\
                \text{飞跃条件:实践+理性思考}
            \end{cases}\\
            \text{第二次飞跃:从认识到实践}\begin{cases}
                \text{认识世界的目的是改造世界}\\
                \text{认识的真理性只有在实践中才能得到检验和发展}\\
                \text{中介环节:确定实践目的、形成实践理念、进行中间实验、运用科学的}\\
                \qquad\text{实践方法等}
            \end{cases}\\
            \text{反复性、无限性}
        \end{cases}\]
\end{itemize}

4.真理\begin{itemize}
    \item 性质\[\begin{cases}
            \text{客观性:内容和标准客观}\\
            \text{绝对性:真理主客观统一的确定性和发展的无限性}\\
            \text{相对性:人们在一定条件下对客观事物及其本质和发展规律的正确认识总是有限度的、不完善的}\\
            \text{真理的绝对性和相对性辩证统一}
        \end{cases}\]
    \item 真理与谬误对立统一,在一定条件下相互转化,坚持真理需修正错误
    \item \textbf{实践是检验真理的唯一标准}\[\begin{cases}
            \text{真理的本性:真理是人们对客观事物及其发展规律的正确反映,它的本性在于主观和客观相符合}\\
            \text{实践的特点:实践具有直接现实性,实践的直接现实性是它的客观实在性的具体表现}\\
            \text{逻辑证明起补重要充作用}
        \end{cases}\]
\end{itemize}

5.价值\begin{itemize}
    \item 价值是反映主体和客体之间意义关系的哲学范畴,是客体对个人、群体乃至整个社会的生活和活动所具有的意义
    \item 基本特性:主体性、客观性、多维性、社会历史性
    \item 价值评价及其特点\[\begin{cases}
            \text{评价以主客体的价值关系为认识对象}\\
            \text{评价结果与评价主体直接相关}\\
            \text{评价结果的正确与否依赖于对客体状况和主体需要的认识}\\
            \text{有科学与非科学之别}
        \end{cases}\]
    \item 价值观是人们关于价值本质的认识以及对人和事物的评价标准、评价原则和评价方法的观点的体系。它与世界观和人生观是一致的。对民族和国家来说,最持久、最深层的力量是全社会共同认可的核心价值观。
    \item 真理和价值在实践中的辩证统一
\end{itemize}

6.认识世界与改造世界\begin{itemize}
    \item \[\begin{cases}
            \text{认识世界:主体能动地反映客体,获得关于事物的本质和发展规律的科学知识,探索和掌握真理}\\
            \text{改造世界:人类按照有利于自己生存和发展的需要,改变事物的现存形式,创造自己的理想世界和生活方式}\\
            \text{认识世界的目的是为了改造世界,而改造世界又包括改造客观世界和改造主观世界}\\
            \text{认识世界和改造世界、改造客观世界和改造主观世界的过程,也就是从必然走向自由的过程}
        \end{cases}\]
    \item 一切从实际出发是马克思主义认识论的根本要求;实事求是是中国共产党思想路线的核心
    \item 坚持守正创新\[\begin{cases}
            \text{守正:坚持实事求是,坚持真理性认识,坚持正确政治方向}\\
            \text{创新:坚持解放思想,破除与客观事物进程不相符合的旧观念、旧模式、旧做法,发现和运用事物的新联系、}\\
            \qquad\text{新属性、新规律,更有效地认识世界和改造世界}\\
            \text{实践创新为理论创新提供不竭的动力源泉}\\
            \text{理论创新为实践创新提供科学的行动指南}
        \end{cases}\]
\end{itemize}

\section{人类社会及其发展规律}

1.社会存在和社会意识\begin{itemize}
    \item 社会存在:指社会的物质生活条件,包括自然地理环境、人口因素和物质生产方式(决定力量)
    \item 社会意识:是社会存在的反映,是社会生活的精神方面
    \item 辩证统一\[\begin{cases}
            \text{社会存在决定社会意识}\\
            \text{社会意识依赖于社会存在}\\
            \text{社会意识的相对独立性}\begin{cases}
                \text{社会意识与社会存在发展的不完全同步性和不平衡性}\\
                \text{社会意识内部各种形式之间的相互影响及各自具有的历史继承性}\\
                \text{社会意识对社会存在的能动的反作用}\\
            \end{cases}
        \end{cases}\]
    \item 唯物史观与唯心史观\[\begin{cases}
            \text{唯物史观:认为社会存在决定社会意识,社会历史发展具有客观规律}\\
            \text{唯心史观:认为社会意识决定社会存在,忽视物质动因和人民群众的作用}
        \end{cases}\]
\end{itemize}

2.社会基本矛盾\begin{itemize}
    \item 生产力与生产关系\[\begin{cases}
            \text{生产力的基本要素}\begin{cases}
                \text{劳动资料(劳动手段)}\\
                \text{劳动对象}\\
                \text{劳动者}\\
            \end{cases}\\
            \text{科学技术日益成为生产发展的决定性因素,是第一生产力}\\
            \text{生产关系}\begin{cases}
                \text{生产资料所有制关系}\\
                \text{生产中人与人的关系}\\
                \text{产品分配关系}\\
            \end{cases}\\
            \text{生产力决定生产关系,生产关系反作用于生产力}\\
            \text{社会发展的根本规律:生产力与生产关系矛盾运动规律}
        \end{cases}\]
    \item 经济基础与上层建筑\[\begin{cases}
            \text{经济基础:一定社会阶段中占统治地位的生产关系的总和}\\
            \text{上层建筑:建立在一定经济基础之上的意识形态以及相应的制度、组织和设施}\\
            \text{辩证统一}\begin{cases}
                \text{经济基础决定上层建筑}\\
                \text{上层建筑反作用于经济基础}\\
                \text{相互作用构成的二者矛盾运动规律,即上层建筑一定要适合经济基础状况的规律}
            \end{cases}
        \end{cases}\]
\end{itemize}

3.人类社会的发展\begin{itemize}
    \item \textbf{交往}:在一定历史条件下的现实的个人、群 体、阶级、民族、国家之间在物质和精神上相互往来、相互作用、彼此联系的活动\[\begin{cases}
            \text{物质交往和精神交往}\\
            \text{人类实践活动的重要组分}\begin{cases}
                \text{促进生产力的发展}\\
                \text{促进社会关系的进步}\\
                \text{促进文化的发展与传播}\\
                \text{促进人的全面发展}
            \end{cases}\\
            \text{世界历史:各民族、国家通过普遍交往,打破孤立隔绝的状态,进入相互依存、相互联系的世界整体化}\\
            \qquad\text{的历史}
        \end{cases}\]
    \item 社会进步\[\begin{cases}
            \text{社会形态从低级到高级的发展}\\
            \text{同一社会形态内部的发展}
        \end{cases}\]
        社会进步促进人的发展、推动人类解放
    \item 社会形态及其性质\[\begin{cases}
            \text{社会形态}\begin{cases}
                \text{经济形态、政治形态、意识形态}\\
                \text{原始社会、奴隶社会、封建社会、资本主义社会和共产主义社会}
            \end{cases}\\
            \text{统一性与多样性}\begin{cases}
                \text{五种社会形态的依次更替,是社会历史运动的一般过程和一般规律,表现了社会形态更}\\
                \qquad\text{替的统一性}\\
                \text{就某一国家或民族的社会发展的历程而言情况复杂,体现了社会形态更替形式的多样性}
            \end{cases}\\
            \text{必然性与选择性}\begin{cases}
                \text{必然性:社会形态依次更替的过程和规律是客观的,其发展的基本趋势是确定不移的}\\
                \text{历史活动的能动性}\to\text{选择性}
            \end{cases}\\
            \text{否定之否定规律}\to\text{前进性与曲折性}
        \end{cases}\]
    \item 文明:人类创造的所有物质成果、精神成果和制度成果的总和,是标志社会进步程度的范畴,反映了人类社会实践活动的积极成果
\end{itemize}

3.社会历史发展的动力\begin{itemize}
    \item 动力系统\[\begin{cases}
            \text{根本动力:社会基本矛盾}\\
            \text{直接动力:阶级斗争}\\
            \text{重要动力:科技}\\
            \text{精神动力:文化}\\
            \text{决定动力:人民群众}
        \end{cases}\]
    \item 社会主要矛盾:社会基本矛盾的具体体现;不是一成不变的,它在一定条件下会发生转化
    \item 阶级斗争\[\begin{cases}
            \text{阶级是一个经济范畴,也是一个历史范畴;阶级的产生、存在和发展是同经济发展过程联系在一起的}\\
            \text{阶级斗争:阶级利益根本冲突的对抗,阶级之间的对立和斗争}\\
            \text{革命:阶级斗争的最高形式,其实质是革命阶级推翻反动阶级的统治,用新的社会制度代替旧的社会}\\
            \qquad\text{制度。要反对改良主义。}\\
            \text{改革:一定社会为了解决社会基本矛盾而对生产关系和上层建筑进行的深刻的改变和革新,它是社会制度}\\
            \qquad\text{的自我调整和完善,是同一种社会形态发展过程中的量变和部分质变,是推动社会发展的又一重要动力。}\\
            \text{马克思主义的阶级分析方法是认识阶级社会的科学方法}
        \end{cases}\]
    \item 科学技术\[\begin{cases}
            \text{科学:对客观世界的认识,反映客观事实和客观规律的知识体系及其相关的活动}\\
            \text{(生产)技术:人类改造自然、进行生产的方法与手段}\\
            \text{科学技术革命}\begin{cases}
                \text{改变生产方式}\begin{cases}
                \text{改变了社会生产力的构成要素}\\
                \text{改变了人们的劳动形式}\\
                \text{改变了社会经济结构,特别是导致了产业结构发生变革}
            \end{cases}\\
                \text{改变生活方式(信息时代)}\\
                \text{促进了思维方式的变革}
            \end{cases}\\
            \text{需要合理的社会制度保障科学技术的正确运用}
        \end{cases}\]
    \item 文化的作用\[\begin{cases}
            \text{为社会发展提供思想指引}\\
            \text{为社会发展提供精神动力}\\
            \text{为社会发展提供凝聚力量}
        \end{cases}\]
    \item 唯物史观(群众史观)\[\begin{cases}
            \text{认为历史的创造者不是个别英雄,而是人民群众}\\
            \text{立足于现实的人及其本质、整体的社会历史过程、社会历史发展的必然性和人与历史关系的不同层次来}\\
            \qquad\text{探究谁是历史的创造者}
        \end{cases}\]
    \item 人民群众在创造历史过程中的决定作用。人民群众是历史的主体,是历史的创造者,是物质财富的创造者,是社会精神财富的创造者,是社会变革的决定力量。
    \item 无产阶级政党\[\begin{cases}
            \text{群众观点:坚信人民群众自己解放自己,全心全意为人民服务,一切向人民群众负责,虚心向群众学习}\\
            \text{群众路线:一切为了群众,一切依靠群众,从群众中来,到群众中去}
        \end{cases}\]
\end{itemize}

4.个人在历史中的作用\begin{itemize}
    \item 分类\[\begin{cases}
            \text{普通人物}\\
            \text{历史人物}\to\text{杰出人物}
        \end{cases}\]
    \item 任何历史人物的出现都体现了必然性与偶然性的统一,要坚持运用历史分析方法和阶级分析方法
    \item 无产阶级领袖不同于以往历史上的杰出人物,因为他们所代表的是历史上最革命、最先进的阶级,他们在革命和建设中发挥了重大作用
    \item 群众、阶级、政党、领袖的关系:群众是划分为阶级的,阶级通常是由政党领导的,政党是由领袖来主持的。群众、阶级、政党、领袖环环相扣、相互依存,构成一个有机整体
\end{itemize}

\section{资本主义的本质及规律}

1.商品经济和价值规律\begin{itemize}
    \item 商品是用来交换、以交换为目的生产的劳动产品;商品经济是以交换为目的而进行生产的经济形势
    \item 商品的价值\[\begin{cases}
            \text{抽象劳动}\to\text{价值:无差别的一般人类劳动(社会属性)}\\
            \text{具体劳动}\to\text{使用价值:满足人们某种需要的属性(自然属性)}\\
            \begin{cases}
            \text{对立:二者相互排斥,不可兼得}\\
            \text{统一:商品必须同时具备使用价值和价值}
        \end{cases}\\
        \text{决定商品价值量的,不是生产商品的个别劳动时间,而是社会必要劳动时间,随劳动生产率变化;}\\
        \text{与简单劳动、复杂劳动有关}
        \end{cases}\]
    \item 货币是在长期交换过程中形成的固定地充当一般等价物的商品
    \item \textbf{价值规律}:商品的价值量由生产商品的社会必要劳动时间决定,商品交换以价值量为基础,按照等价交换的原则进行\[\begin{cases}
            \text{商品的价格受供求关系影响,围绕价值自发波动}\\
            \text{积极作用:自发调节生产资料和劳动力在社会各生产部门之间的分配比例、刺激社会生产力发展、}\\
            \qquad\text{调节社会收入分配}\\
            \text{消极后果:导致资源浪费、阻碍技术进步、引起收入两极分化}
        \end{cases}\]
\end{itemize}

2.以私有制为基础的商品经济的基本矛盾:私人劳动与社会劳动的矛盾;在资本主义制度下发展为生产社会化与生产资料资本主义私人占有的基本矛盾,决定了资本主义被社会主义取代的必然性。

3.马克思劳动价值论\[\begin{cases}
            \text{扬弃了英国古典政治经济学的观点,为剩余价值理论的创立奠定了基础}\\
            \text{揭示了私有制条件下商品经济的基本矛盾,为从物与物的关系背后揭示人与人的关系提供了理论依据}\\
            \text{揭示了商品经济的一般规律,对理解社会主义市场经济具有指导意义}
        \end{cases}\]

4.资本主义经济制度的产生\[\begin{cases}
            \text{前资本主义社会形态的演进和更替:原始社会、奴隶社会和封建社会}\\
            \text{资本主义生产关系产生:文艺复兴}\\
            \text{资本的原始积累:以暴力手段使生产者与生产资料相分离,货币资本迅速集中于少数人手中,资本}\\
            \qquad\text{主义得以迅速发展}\\
            \text{资本主义所有制的确立:资本家拥有生产资料的所有权,劳动者与生产资料相分离}
        \end{cases}\]

5.劳动力与资本\begin{itemize}
    \item 资本总公式:$G-W-G'$,G为作为资本的货币,W为商品,G'为价值增值后的货币,实现这种价值增值的条件是劳动力成为商品。
    \item 劳动力成为商品的基本条件\[\begin{cases}
            \text{劳动者在法律上是自由人,能够把自己的劳动力当做自己的商品来支配}\\
            \text{劳动者没有任何生产资料,没有生活资料来源,因而不得不依靠出卖劳动力为生}
        \end{cases}\]
    \item 劳动力商品的使用价值是价值的源泉,在消费过程中能够创造新价值,而且这个新的价值比劳动力本身的价值更大;货币所有者购买到劳动力以后,还得到一个增殖的价值,即剩余价值(m)。
\end{itemize}

6.\textbf{剩余价值的生产与资本积累}\begin{itemize}
    \item 资本主义生产过程\[\begin{cases}
            \text{物质资料的生产过程}\\
            \text{剩余价值的生产过程(价值增值过程),}\\
            \text{此过程中}\begin{cases}
                \text{必要劳动时间}t'\text{,用于再生产劳动力的价值}\\
                \text{剩余劳动时间}t\text{,用于无偿地为资本家生产剩余价值}m
            \end{cases}
        \end{cases}\]
    \item 数值关系\[\begin{cases}
            \text{全部预付资本}\begin{cases}
                \text{不变资本}c\text{:以生产资料形态存在的资本}\\
                \text{可变资本}v\text{:用来购买劳动力的资本}
            \end{cases}\\
            \text{商品价值}W=c+v+m\\
            \text{新价值}=v+m\\
            \text{资本有机构成}=\dfrac{c}{v}\\
            \text{利润率}=\dfrac{m}{c+v}\\
            \text{剩余价值率(剥削率)}m'=\dfrac{m}{v}=\dfrac{t'}{t}
        \end{cases}\]
    \item 剩余价值的生产\[\begin{cases}
            \text{绝对剩余价值:在必要劳动时间不变的条件下,由于延长工作日的长度而生产的剩余价值}\\
            \text{相对剩余价值:在工作日长度不变的条件下,通过缩短必要劳动时间而相对延长剩余劳动时间生产的剩余价值}
        \end{cases}\]
    \item 资本积累\[\begin{cases}
            \text{简单再生产:在原有规模的基础上重复进行}\\
            \text{扩大再生产(资本积累):将部分剩余价值转化为资本,在扩大的规模上重复进行}\\
            \to\begin{cases}
                \text{社会财富两极分化}\\
                \text{导致资本有机构成​}c/v \text{提高,造成相对过剩人口}
            \end{cases}\\
            \text{历史趋势:资本积累使生产社会化与资本主义私人占有制之间的基本矛盾日益尖锐,最终资本主义必然灭亡}
        \end{cases}\]
    \item 资本的循环周转\[\begin{cases}
            \text{产业资本的三个阶段:购买阶段、生产阶段、售卖阶段;对应职能:货币、生产、售卖资本的职能}\\
            \text{产业资本的运动的前提}\begin{cases}
            \text{产业资本的三种职能形式必须在空间上同时并存}\\
            \text{产业资本的三种职能形式必须在时间上继起}
        \end{cases}\\
            \text{资本的周转:资本周而复始、不断反复的循环}\\
            \text{周转快慢}\begin{cases}
                \text{资本周转的时间}\\
                \text{生产资本的固定资本和流动资本的构成}
            \end{cases}
        \end{cases}\]
\end{itemize}

7.经济危机\begin{itemize}
    \item 资本主义基本矛盾:生产资料资本主义私人占有和生产社会化之间的矛盾
    \item 经济危机\[\begin{cases}
        \text{本质特征:生产相对过剩}\\
        \text{可能性:由货币作为支付手段和流通手段引起的}\\
        \text{根本原因}\begin{cases}
            \text{生产无限扩大的趋势与劳动人民有支付能力的需求相对缩小的矛盾}\\
            \text{个别企业内部生产的有组织性和整个社会生产的无政府状态之间的矛盾}
        \end{cases}\\
        \text{周期性:由资本主义基本矛盾运动阶段运动的特点决定}
    \end{cases}\]
\end{itemize}

8.经济全球化
\begin{itemize}
    \item 定义:在生产不断发展、技术加速进步、社会分工和国际分工不断深化、生产的社会化和国际化程度不断提高的情况下,世界各国、各地区的经济活动越来越超出某一国家和地区的范围而相互联系、相互依存的过程
    \item 表现\[\begin{cases}
        \text{生产全球化}\\
        \text{贸易全球化}\\
        \text{金融全球化}
    \end{cases}\]
    \item \textbf{影响}\[\begin{cases}
            \text{积极}\begin{cases}
            \text{为发展中国家提供先进技术和管理经验}\\
            \text{为发展中国家提供更多的就业机会}\\
            \text{推动发展中国家国际贸易的发展}
        \end{cases}\\
            \text{消极}\begin{cases}
                \text{发达国家与发展中国家在经济全球化的进程中的地位和收益不平等、不平衡}\\
                \text{加剧了发展中国家资源短缺和环境污染}\\
                \text{一定程度上增加了经济风险}
            \end{cases}
        \end{cases}\]
\end{itemize}

9.二战后资本主义的变化(垄断资本形式的变化)
\begin{itemize}
    \item \textbf{表现}(?)\[\begin{cases}
        \text{生产资料所有制的变化}\\
        \text{劳资关系和分配关系的变化}\\
        \text{社会阶层和阶级结构的变化}\\
        \text{经济调节机制和经济危机形态的变化}\\
        \text{政治制度的变化}
    \end{cases}\]
    \item 原因\[\begin{cases}
        \text{科学技术革命和生产力发展}\\
        \text{工人阶级争取自身权利和利益的斗争}\\
        \text{社会主义制度初步显示的优越性}\\
        \text{改良主义政党对资本主义制度的改革}
    \end{cases}\]
\end{itemize}

\section{第六、七章重点内容}

1.\textbf{科学社会主义基本原则}\begin{itemize}
    \item 资本主义必然灭亡,社会主义必然胜利
    \item 无产阶级是最先进最革命的阶级,肩负着推翻资本主义旧世界、建设社会主义和共产主义新世界的历史使命
    \item 无产阶级革命是无产阶级进行斗争的最高形式,以建立无产阶级专政的国家政权为目的
    \item 要在生产资料公有制基础上组织生产,以满足全体社会成员的需要为生产的根本目的
    \item 要对社会生产进行有计划的指导和调节,实行按劳分配原则
    \item \textbf{要合乎自然规律地改造和利用自然,努力实现人与自然的和谐共生}
    \item 必须坚持科学的理论指导,大力发展社会主义先进文化
    \item 无产阶级政党是无产阶级的先锋队,社会主义事业必须始终坚持无产阶级政党的领导
    \item 社会主义社会要大力解放和发簪生产力,逐步消灭剥削和消除两极分化,实现共同富裕和社会全面进步,并最终向共产主义社会过渡
    \item 共产主义是人类最美好的社会制度,实现共产主义是共产党人的最高理想
\end{itemize}

2.\textbf{共产主义社会的基本特征}\begin{itemize}
    \item 物质财富极大丰富,消费资料按需分配
    \item 社会关系高度和谐,人们精神境界极大提高
    \item 实现每个人自由而全面的发展,人类从必然王国向自由王国的飞跃
\end{itemize}

3.\textbf{社会主义建设过程的长期性}\begin{itemize}
    \item 生产力发展状况的制约
    \item 经济基础和上层建筑发展状况的制约
    \item 国际环境的严峻挑战
    \item 马克思主义执政党对社会主义发展道路的探索和对社会主义建设规律的认识,需要一个长期的过程
\end{itemize}

\end{document}